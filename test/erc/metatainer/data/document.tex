\documentclass[nojss,shortnames]{jss}
\usepackage[utf8]{inputenc}

\providecommand{\tightlist}{%
  \setlength{\itemsep}{0pt}\setlength{\parskip}{0pt}}

\author{
Ted Tester\\University of Munster \And Carl Connauthora\\N.O.N.E
}
\title{}
\Keywords{lorem, ipsum, dolor, sit, amet}

\Abstract{
Tempus eget nunc eu, lobortis condimentum nulla. Nam sagittis massa nec
libero luctus facilisis. Suspendisse ac ornare ligula. Morbi non
dignissim sem. Pellentesque eleifend neque nec dui interdum varius.
}

\Plainauthor{Ted Tester, Carl Connauthora}
\Plainkeywords{lorem, ipsum, dolor, sit, amet}

%% publication information
%% \Volume{50}
%% \Issue{9}
%% \Month{June}
%% \Year{2012}
\Submitdate{}
%% \Acceptdate{2012-06-04}

\Address{
    Ted Tester\\
  University of Munster\\
  This might be a correct address Yet it might also be lorem
  ipsum\newline Another line 252b\newline 00000 Anywhere, Nomansland\\
  E-mail: \href{mailto:ted.tester@awebsite8372930.org}{\nolinkurl{ted.tester@awebsite8372930.org}}\\
  URL: \url{http://404.awebsite8372930.org}\\~\\
    }

\usepackage{amsmath} \usepackage{array} \usepackage{caption}
\usepackage{subcaption} \usepackage{float} \usepackage{framed}
\usepackage{listings}

\begin{document}

\section{test2 rmd for metaextract
demo}\label{test2-rmd-for-metaextract-demo}

\section{And here comes some Text}\label{and-here-comes-some-text}

Computer software, or simply software, is that part of a computer system
that consists of encoded information or computer instructions, in
contrast to the physical hardware from which the system is built. The
term ``software'' was first proposed by Alan Turing and used in this
sense by John W. Tukey in 1957. In computer science and software
engineering, computer software is all information processed by computer
systems, programs and data. Computer software includes computer
programs, libraries and related non-executable data, such as online
documentation or digital media.

Computer hardware and software require each other and neither can be
realistically used on its own. At the lowest level, executable code
consists of machine language instructions specific to an individual
processor---typically a central processing unit (CPU). A machine
language consists of groups of binary values signifying processor
instructions that change the state of the computer from its preceding
state. For example, an instruction may change the value stored in a
particular storage location in the computer---an effect that is not
directly observable to the user. An instruction may also (indirectly)
cause something to appear on a display of the computer system---a state
change which should be visible to the user. The processor carries out
the instructions in the order they are provided, unless it is instructed
to ``jump'' to a different instruction, or interrupted. The majority of
software is written in high-level programming languages that are easier
and more efficient for programmers, meaning closer to a natural
language.{[}1{]} High-level languages are translated into machine
language using a compiler or an interpreter or a combination of the two.
Software may also be written in a low-level assembly language,
essentially, a vaguely mnemonic representation of a machine language
using a natural language alphabet, which is translated into machine
language using an assembler.

\begin{CodeChunk}
\begin{CodeOutput}

Attaching package: 'dplyr'
\end{CodeOutput}
\begin{CodeOutput}
The following objects are masked from 'package:stats':

    filter, lag
\end{CodeOutput}
\begin{CodeOutput}
The following objects are masked from 'package:base':

    intersect, setdiff, setequal, union
\end{CodeOutput}
\begin{CodeOutput}

-----------------------------------------------------------
PBS Mapping 2.69.76 -- Copyright (C) 2003-2016 Fisheries and Oceans Canada

PBS Mapping comes with ABSOLUTELY NO WARRANTY;
for details see the file COPYING.
This is free software, and you are welcome to redistribute
it under certain conditions, as outlined in the above file.

A complete user guide 'PBSmapping-UG.pdf' is located at 
/home/daniel/R/x86_64-pc-linux-gnu-library/3.3/PBSmapping/doc/PBSmapping-UG.pdf

Packaged on 2015-04-23
Pacific Biological Station, Nanaimo

All available PBS packages can be found at
http://code.google.com/p/pbs-software/

To see demos, type '.PBSfigs()'.
-----------------------------------------------------------
\end{CodeOutput}
\begin{CodeOutput}
[1] "WBTMKALNESGICQZJ"
\end{CodeOutput}
\begin{CodeOutput}
[1] 16
\end{CodeOutput}


\begin{center}\includegraphics{document_files/figure-latex/unnamed-chunk-1-1} \end{center}

\begin{CodeOutput}
[1] "tested 2"
\end{CodeOutput}
\end{CodeChunk}

Computer software, or simply software, is that part of a computer system
that consists of encoded information or computer instructions, in
contrast to the physical hardware from which the system is built. The
term ``software'' was first proposed by Alan Turing and used in this
sense by John W. Tukey in 1957. In computer science and software
engineering, computer software is all information processed by computer
systems, programs and data. Computer software includes computer
programs, libraries and related non-executable data, such as online
documentation or digital media.



\end{document}

